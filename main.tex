%Time-stamp: "Last modified: 2020-09-15 16:07:32 (d_yasaki)"
\documentclass[ms]{uncgdissertationexp2}
% default is 12pt, phd, doublespaced.
% Masters students should use the ma on as shown below.
% \documentclass[ma]{uncgdissertation}

%%------------------------------------------------------------------%%
%%------------------------- Import Packages ------------------------%%
%%------------------------------------------------------------------%%
%% This is where you can put other packages that you may need. 
\usepackage{microtype, amsmath, amsfonts, amsthm, graphicx, booktabs}
\usepackage[colorlinks=false]{hyperref}
\usepackage[math]{blindtext} %Can remove this and all
%% references to \blinddocument and \blindmathpaper
%% to eliminate the Lorem ipsum



% \usepackage{showframe} 
% useful package to ensure margins are correct.
%%------------------------------------------------------------------%% 
%%--------------------------- Content ------------------------------%%
%%------------------------------------------------------------------%%
%% Members of committee.  Guidelines say don't use Dr.
%% Masters students are required to have chair plus two
%% PhD students require chair plus three.
%% The class can handle up to chair plus five.
\chair{Stephen R. Tate}
%\chair{Jing Deng}
\member{Chun Jiang Zhu}
\member{Insa R. Lawler}   
%\member{Leonhard Euler}   

%% Your name goes here.
% \student{Firstname}{Lastname} 
%% Some other options
\student{Larry Joshua}{Crotts}  % a full middle name
%\student{Joe M.}{Schmoe}       % a middle initial
 
%% Thesis Title
%%    +  Capitalize first letter of important words. 
%%    +  Use inverted pyramid shape if title spans more than one line.
%%  Note: You can force break the title onto multiple lines using
%%  \break instead of \\. 
\title{Construction and Evaluation of a Gold Standard Syntax for Formal Logic Formulas and Systems}

%% Degree year.    
\degreeyear{2022}

%%------------------------------------------------------------------%% 
%%----------------------- Personal Macros --------------------------%%
%%------------------------------------------------------------------%%
%% A central location to add your favorite macros.  A few examples are
%% given below.  See tips for samples.

%% In order to get singlespacing, uncomment the line below.
%\renewcommand{\doublespacing}{\singlespacing}
\doublespacing
%% Theorem, Lemma, etc. environments.  You can rename if you wish.                    
% Theorem style and numbering convention                                              
\theoremstyle{plain}
\newtheorem{theorem}{Theorem}[chapter]
\newtheorem{lemma}[theorem]{Lemma}
\newtheorem{proposition}[theorem]{Proposition}
\newtheorem{conjecture}[theorem]{Conjecture}
\newtheorem{corollary}[theorem]{Corollary}
\newtheorem{algorithm}[theorem]{Algorithm}

% Definition type object style and numbering convention                               
\theoremstyle{definition}
\newtheorem{definition}[theorem]{Definition}
\newtheorem{example}[theorem]{Example}

% Remark type object style and numbering                                              
\theoremstyle{remark}
\newtheorem*{remark}{Remark}  % the star makes them not numbered                      
\newtheorem*{notation}{Notation}
\newcommand{\titlecaption}[2]{\caption[#1]{#1. #2}}

%% Other macros
\newcommand{\ZZ}{\mathbb{Z}}  % Integers
\newcommand{\XX}{\mathfrak{X}}  
%%------------------------------------------------------------------%%
\pagestyle{plain} % Eliminates running headers 

% Section styling.
\titleformat*{\section}{\normalsize\bfseries}
\titleformat*{\subsection}{\normalsize\itshape}

\begin{document}
\frontmatter      % required

%%------------------------------------------------------------------%%
%% -------------------------- Abstract -----------------------------%%
%%------------------------------------------------------------------%%
\begin{abstract}
The abstract page is a required component of the thesis/dissertation.
The abstract should be a brief summary of the paper, stating only the
problem, procedures used, and the most significant results and
conclusions. Explanations and opinions are omitted. Remember to
include the necessary information regarding any multimedia components
included in the document. The abstract must be approved by your
advisor/committee chair. 
\end{abstract}

%%------------------------------------------------------------------%%
%%---------------------------- Title page --------------------------%%
%%------------------------------------------------------------------%%
%% The title page is required. 
\maketitlepage  

%%------------------------------------------------------------------%%
%% ------------------------ Copyright page -------------------------%%
%%------------------------------------------------------------------%%
%% This page is required if you opt for a copyright.  Otherwise, don't
% include it.  To omit, just comment out the line below.
\makecopyrightpage

%%------------------------------------------------------------------%%
%%---------------------------- Dedication --------------------------%%
%%------------------------------------------------------------------%%
\begin{dedication}
 To my Viola.
\end{dedication}

%%------------------------------------------------------------------%%
%%------------------------ Approval page  --------------------------%%
%%------------------------------------------------------------------%%
%% The approval page is required.  If all of your infomation is entered
%% correctly in the contents section, this should come out correctly.
\makeapprovalpage

%%------------------------------------------------------------------%%
%%-------------------------- Acknowledgements ----------------------%%
%%------------------------------------------------------------------%%
%% The acknowledgements are optional but highly recommended.  See tips
%% for details. 
\begin{acknowledgments}
I would like to extend my gratitude and thanks to the esteemed Dr. Steve Tate for not only overseeing and advising this thesis, but also for being a fantastic mentor and professor throughout my time at UNC Greensboro. I sincerely appreciate Dr. Nancy Green for introducing me to the wonderful world of academic-level computer science research, as well as Dr. Insa Lawler in the Philosophy department for introducing me to the exciting adventure that is formal logic and its pedagogical impact. Their unwavering guidance, mentorship, and insight influenced me to pursue graduate school.

Outside of UNCG, I thank my parents for their love and support throughout my education. I also thank two of my best friends: Audree Logan and Andrew Matzureff, for their support and friendship from high school to now.

It also cannot go without saying that I am forever grateful for the support from my loving fiancée Viola. You never let me down.

Finally, I am deeply indebted to Mr. Tony Smith: my former Advanced Placement\footnote{https://ap.collegeboard.org/} Computer Science teacher. Without him, I would not be where I am now. Thank you for seeing (and helping me reach) my potential.
\end{acknowledgments}

%%------------------------------------------------------------------%%
%%----------------------------- Preface ----------------------------%%
%%------------------------------------------------------------------%%
%% The preface is optional.
\begin{preface}
The basis for this research stems from my love of teaching. When I took my first introduction to formal logic course as an undergraduate, I was taken aback by its amazing appeal and relation to computer science. From my semesters serving as a tutor/teaching assistant in the Philosophy department at UNC Greensboro, I saw many students that struggled with this material. The problems ranged from its confusing syntax, proof techniques, and esoteric notation. At that time, I thought to myself, "Why not make a tool that helps students understand it better?" Of course, that question had already been answered and deeply investigated across multiple disciplines, but I knew that there had to be more. Once I began my exploration, I quickly realized that online solvers, theorem provers, proof assistants, and similar tools do not have a ubiquitous input format, and testing their algorithms was far more cumbersome than I initially expected. This evolved into the desire for a gold standard syntax for both zeroth and first order logic systems.
\end{preface}


%%------------------------------------------------------------------%%
%%---------------------- Table of Contents -------------------------%%
%%------------------------------------------------------------------%%
%% The table of contents is required.  
\tableofcontents 

%%------------------------------------------------------------------%%
%%---------------------- List of Tables ----------------------------%%
%%------------------------------------------------------------------%%
% Recommended if you have tables.  Comment out if you don't have
% tables. 
%\listoftables   


%%------------------------------------------------------------------%%
%%---------------------- List of Figures ---------------------------%%
%%------------------------------------------------------------------%%
% Recommended if you have figures.  Comment out if you don't have
% figures. 
%\listoffigures   


%%------------------------------------------------------------------%%
%% This signifies that you are done with the frontmatter and ready to
%% proceed to the main part.  The rest of your document goes below.
\mainmatter % required
%%------------------------------------------------------------------%%
\chapter{Introduction}
    \section{Overview}
    Formal logic, otherwise known as classical formal logic, is a subset of philosophy that branches into other related disciplines such as computer science, statistics, mathematics, and similar sciences. Logic, however, is taught in non-science fields like communicative studies primarily to reinforce critical thinking and improve deductive skills for argumentation. Per Stanford's Encyclopedia on Classical Logic, logic is a tool used for studying correct reasoning. Its existence spawned questions ranging from its use in mathematics as an aid to disambiguate problems and proofs to considering it as an extension to natural language \cite{stanfordencyclopedia}. As Hatcher \cite{hatcher} states, due to the increased viewing of rhetoric and opinion versus factual knowledge in modern media across television, social media, and other such mediums, the need for strong logical thinking abilities is crucial for evaluating, analyzing, and debating arguments and claims. Hatcher, likewise, mentions that standard logical deductive forms such as methods of inference and syllogisms serve as critical components for a student's ability to determine the validity of an argument and the relation (or lack thereof) of premises to conclusions. A desire for valid and sound arguments from students constitutes and contributes to a wider adoption of formal logic classes in universities, or at the very least, the pedagogy of invalid arguments with how to refute incorrect and, sometimes egregious, contentions.  Formal logic's relation to computer science, in particular, ... \textbf{talk about how we can use formal logic for mathematical proofs, Boolean logic for circuitry, set theory, etc.}
    \section{Contribution}
    \section{Thesis Content}
    \section{Terminology}
    Before we continue, we will define some terms frequently used in formal logic-related work.
    \begin{definition}[Well-Formed Formula]
    \end{definition}
    
    \begin{definition}[Proposition]
    \end{definition}

    \begin{definition}[Proof]
    \end{definition}
    
    \begin{definition}[Theorem]
    \end{definition}
\chapter{Related Work}
    In this chapter, we will discuss the related work and prior contributions to the discipline of natural deduction pedagogy, as well as efforts to modernize and increase its effectiveness for students with a weaker background in, for example, mathematics. 
    Extending formal logic to a technological education is not a new idea---there exist many online solvers, provers, and programming languages designed to suit the needs of logic students, or those that use formal logic in some manner. We will also mention more powerful theorem provers that are aimed at experts/more experienced users.
    \section{Formal Logic Tutors}
        \subsection{Propositional Logic}
        Propositional logic, also known as zeroth-order logic (or in other disciplines as sentence logic, sentential logic, Boolean logic, combinatorial logic, or propositional calculus), according to Hein \cite{hein}, is a language of propositions that conform to rules. Propositional logic is comparatively simpler than first-order predicate logic described in section II.1.2---it does not use variables, constants, or quantifiers of any kind. Rather, in this language, there are four binary (two-place/two-arity) connectives: logical conjunction, logical disjunction, logical implication, and the biconditional, as well as one unary (one-place/one-arity) operator: logical negation. \textbf{Show a table and describe different notation by different authors?}
        
        Because of the reducible nature of propositional logic to simple structures and representations, there exist plentiful online truth table generators that provide detailed and immediate feedback for users while solving problems and well-formed formulas. Further, such generators work well not only for formal logic, but also computer science, mathematics, and electrical engineering, allowing students to enter a Boolean truth value (i.e., true/false) for an operand or proposition and the computer will determine if it is valid or invalid for an arbitrary cell. 
%         Automatic logic tutors and theorem provers exist in many dimensions and formats, ranging from downloadable and executable software to modern and lively web applications. From our investigations, however, these systems and software often do not provide a beginner-friendly experience, nor do they provide the functionality we want students to engage with. Others like Near et al. \cite{near} introduce fast theorem provers written in functional programming languages, but their broad intention is not to teach students, particularly non-computer science students.

% For starters, there exist plenty of online truth table generators that work well not only in the formal logic domain, but also electrical engineering, computer science, and (discrete) mathematics domains. Some even provide immediate feedback for the user as they attempt to derive the truth table by hand \cite{truthtablefennell}. An apparent drawback is that they require a student to have prior experience with the underlying logic or preexisting knowledge of entering values into a truth table \cite{koedinger}. Beyond this, Lukins et al. \cite{lukins} described and built the P-Logic Tutor system for propositional logic: a Java Web Start (JNLP) system. Today, their provided link is offline, so there is no way of evaluating or testing its functionality compared to its more modern counterparts. From the details the authors provide, though, students could enter their own data into the program and receive feedback on its correctness. One significant downside to the P-Logic Tutor is that it only covers/handles propositional logic across all its units and tools, as its name suggests. Moreover, its usage required students to log in for purposes of improving and personalizing the experience, a mandate that other systems lack. Requirements like this dissuade users from the tool who are not affiliated with their university. Another software-based solution (i.e., executable outside the browser) is LEGEND by Vlist \cite{vlist}. LEGEND is untestable as it is closed-source and unavailable to the public, but it allows the user to prove and generate proofs from a (simple) given propositional formula. Cerna et al. \cite{Cerna2019AMA} developed \textbf{AX}olotl: a clean Android formal logic tutor which includes several types of proofs and tutorials for deriving examples, though its reliance on a file protocol to load examples is a bit cumbersome for the non-savvy student or instructor. Further, it appears to focus heavier on an accelerated natural deduction curriculum, whereas FLAT attempts to target absolute beginners at the material. Almost all systems we investigated only allow for propositional logic proofs or evaluation because of first-order predicate's infinite nature when applying universal quantifier rules as well as the general difficulty curve over propositional logic.
        \subsection{First-Order Logic}
        \subsection{Problem/Solution Generators}
        Ahmed et al. wrote.........
        Hladik...
        Amendola...
%         Because natural deduction has close ties with discrete math, computer science, and phi-
% losophy, its appearance in online solvers is to be expected. To our surprise, however,
% there were not many propositional logic natural deduction generators available, and
% even less so for first-order predicate logic. To determine the effectiveness of FLAT’s
% natural deduction algorithm, we compared its efficiency (measured in number of lines
% in the generated proof) to three different systems freely available online: TAUT from
% the Buenos Aires Logic Group [3], NaturalDeduction from Jukka Häkkinen [14], and
% Natural Deduction from the Grenoble Computer Science Laboratory [8]. Our test suite
% consisted of 52 propositional logic formulas varying in complexity. All 52 were de-
% ductively valid arguments with some requiring only one line to deduce the necessary
% conclusion5. We have discovered that many solvers use either an indirect or conditional
% proof approach to solving natural deduction problems. FLAT does not currently handle
% conditional proofs (with nested subproofs), so any problem that requires a conditional
% proof is unsolvable. None of the examples in our test suite, however, required the use of
% a subproof. FLAT uses several syllogisms and axioms to search for sub-goals that other
% systems manually derive which slows computation time and over-complicate the proof.
% We also found that some systems did not allow the use of certain symbols or input, such
% as the biconditional operator, uppercase propositions, or arbitrary letter propositions (re-
% quiring us to alter our test cases for these systems). Figure 2 shows a comparison of the
% four systems using the metric described above. Some test cases generate a line count of
% 0. This indicates that the test was unsolvable in that system due to a symbolic restriction
% or resource limitation.
        LLAT... Graham defines a semantic tableau...
    \section{Automatic Theorem Provers}
    Coq..., $\alpha$\textsf{lean}TAP,... 
    \section{Boolean Satisfiability Solver Input Formats} 
\chapter{Methods}
    In this chapter, we explain our evaluation method and metrics for assessing three publicly-available natural deduction systems against our prover. Additionally, we construct a formal definition for a standardized and uniform syntax for writing and, more importantly, testing differing logic systems and algorithms.
    \section{Evaluation of Natural Deduction Systems}
    \section{Gold Standard for Formal Logic System Syntax}
\chapter{Results and Discussion}
\chapter{Conclusion and Future Direction}
%\chapter{Templates}

% The Graduate School is fairly flexible with thesis formatting, but we do require use of our templates to help guide candidates in the right direction. The highest priorities of formatting theses and dissertations are self-consistency and disciplinary norms. Text should be formatted into a single column using US Letter paper with portrait orientation. If necessary, tables may be formatted onto a page with landscape orientation. 

% \section{Acceptable types of theses and dissertations}

% Three types of acceptable theses and dissertations
% \begin{enumerate}
% \item Traditional Thesis/Dissertation. The standard document in many disciplines contains five chapters, which may consist of an introduction, literature review, methods, results, and conclusions; however, neither the number nor the topics of these chapters are set by these guidelines. 
% \item Article-based Thesis/Dissertation. In such a work, a collection of research articles that will be submitted, have been submitted, have been accepted, or have been published in discipline-specific academic journals or outlets comprise the main chapters. Students choosing this option should work with the publisher and must consult UNCG's policy on use of published work in a thesis or dissertation. Such work shall include uniform front matter and an introductory chapter(s) providing broader context and impact of the work. A conclusion chapter may be included.
% \item Creative Work. Upon the recommendation of programs, the Graduate School may accept theses and dissertations in alternative formats appropriate to the discipline. Such work must include uniform front matter.
% \end{enumerate}

% \section{Organization and Required Elements}

% The following items are required regardless of the format of the document.

% \begin{enumerate} \item Frontmatter --- generated by template
% \begin{enumerate}
% 	\item Abstract (required)
% 	\item Title page (required)
% 	\item Copyright page (required if seeking copyright registration)
% 	\item Dedication page (optional)
% 	\item Approval page (required)
% 	\item Acknowledgements (optional)
% 	\item Preface (optional)
% 	\item Table of contents (required, must be automatically generated and include references to chapters, subheadings, etc)
% 	\item List of figures (optional)
% 	\item List of tables (optional)\end{enumerate}
% \item Body of dissertation --- must fit into one of the three styles described above.
% \item References or Bibliography --- must indicate all works cited.
% \begin{enumerate}
% 	\item Student should use a commonly accepted style in the discipline.
% 	\item For article-based dissertations, individual articles? bibliographies should be replaced with a single comprehensive list at the end.
% \end{enumerate}
% \item End matter --- optional
% \begin{enumerate}
% 	\item Appendices (grouped by category), which may include tables of data, code, forms, etc.
% 	\item Biographical sketch.
% \end{enumerate}\end{enumerate}
% \blinddocument
% \section{Some math}
% \blindmathpaper \cite{A}

% \begin{table}
%  \caption{Meaning of Street Light Colors}   
%  \label{tab:lights}
% \centering
% \begin{tabular}{c c}
%  \toprule
%  Colors & Meaning\\
%  \midrule
%  Red  & Stop \\
%  Green & Go \\ 
%  Yellow & Speed up\\
%  \bottomrule
%  \end{tabular}
%  \end{table}

% \blindmathpaper \cite{grun.book}

% \begin{figure}
% \centering
% \includegraphics[width=0.65\textwidth]{logo}
% \titlecaption{Spartan Logo}{The UNCG Spartan Logo is show here at 0.65
% of the text width.}
% \label{fig:logo}
% \end{figure}
% \blindmathpaper

% \blinddocument

%%------------------------------------------------------------------%%
%%------------------------ Bibliography ----------------------------%%
%%------------------------------------------------------------------%%
%% Replace the myreferences with the name of your bib file.  Then you
%% can run bibtex as usual.
%%------------------------------------------------------------------%%


\bibliography{myreferences}
\bibliographystyle{plain}

%%------------------------------------------------------------------%%
%%------------------------- Appendices -----------------------------%%
%%------------------------------------------------------------------%%
%% If you choose not to have appendices, comment out the \appendix
%% line and the chapters below.
%%------------------------------------------------------------------%%
% \appendix
% \chapter{Sample appendix}
% \blindmathpaper

%%------------------------------------------------------------------%%
\backmatter
%%------------------------------------------------------------------%%
%%----------------------- YOU ARE FINISHED ! -----------------------%%
%%------------------------------------------------------------------%%
\end{document}
